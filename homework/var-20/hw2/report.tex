% !TEX root = main.tex

\section*{Условие} 
После n = 240 бросков игральной кости ``шестерка'' выпала 75 раз. При уровне значимости \(\alpha\) = 0.1 проверить гипотезу о том, что кость правильная.

\section*{Решение} 

n = 240 \\
k = 75 \\
\(\alpha\) = 0.1 \\

Мат. ожидание для идеальной игральной кости:
\begin{equation*}
	\mu_0 = \frac{1}{6}
\end{equation*}

Мат. ожидание нашей кости:
\begin{equation*}
	\mu = \frac{k}{n} = \frac{75}{240}
\end{equation*}

Дисперсия для идеальной кости:
\begin{equation*}
	\sigma^2 = p*q = \frac{1}{6}*\frac{5}{6} = \frac{5}{36}
\end{equation*}

Гипотезы:
\begin{align*}
	H_0 &= \{ \mu = \mu_0 \} \text{ -- Основная гипотеза}, \\
	H_1 &= \{ \mu \not= \mu_0 \} \text{ -- Конкурирующая гипотеза}
\end{align*}

Статистика и ее закон распределения:
\begin{equation*}
	T(\vec{x_n}) = \frac{\mu_0 - \bar{x}}{\sigma}\sqrt{n} \sim N(0, 1)
\end{equation*}

Условие, определяющее критическую область W:
\begin{gather*}
	W: \{ \vec{x_n}: \abs{T(\vec{x_n})} \ge U_{1 - \frac{\alpha}{2}} \}, \\
	U_{1 - \frac{\alpha}{2}} = U_{1 - 0,05} = U_{0,95} = 1,645
\end{gather*}

\begin{gather*}
	T(\vec{x_n}) = \frac{ \frac{1}{6} - \frac{75}{240} }{ \sqrt{\frac{5}{36}} } \sqrt{240} = \\ 
	= \frac{ -35 * 6 * \sqrt{240} }{ 240 \sqrt{5} } = 
	- \frac{7\sqrt{5} * 6}{ \sqrt{240} } \simeq -6,062 * \abs{-6,062} \ge 1,645 
\end{gather*}

\section*{Ответ} 
Кость неправильная.