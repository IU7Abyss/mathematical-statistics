% !TEX root = main.tex

\section{Задача №2. Метод моментов}

\paragraph{Условие.} С использованием метода моментов для выборки $\vec{x}_n = (x_1, \dots, x_n)$ найти точечные оценки параметров заданного закона распределения генеральной совокупности $X$.

\[
    f_X (x) = \frac{1}{2^{\theta/2} \cdot \textsf{Г} (\theta/2)} x^{\frac{\theta}{2} - 1} e^{-\frac{x}{2}}, \quad x > 0\,.
\]

\paragraph{Решение.}\hfill\\

\noindent
Рассмотрим закон распределения c неизвестынм параметром $\theta$

\[
    f_X (x; \theta) = \frac{1}{2^{\theta/2} \cdot \textsf{Г} (\theta/2)} x^{\frac{\theta}{2} - 1} e^{-\frac{x}{2}}, \quad x > 0\,.
\]

\noindent
Это не что иное, как \emph{$\chi^2$-распределение с $\theta$ степенями свободы}. Отметим, что $\chi^2$-распре-деление является частным случаем \emph{гамма-распределения с параметром формы $\alpha = \theta/2$ и параметром масштаба $k = 2$}. Таким образом

\[
    m_1(\theta) = \Expect X = \alpha k = \frac{\theta}{2} \cdot 2 = \theta \;\; = \;\; \overline{X}_n = \hat{m}_1(\overline{X}_n).
\]

\paragraph{Ответ.} $\hat\theta = \overline{X}_n$, при $x > 0$.