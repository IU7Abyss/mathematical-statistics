% !TEX root = main.tex

\section{Задача №1. Предельные теоремы теории вероятностей}

\paragraph{Условие.} В Москве рождается в год около $122500$ детей. Считая вероятность рождения мальчика равной $0.51$, найти вероятность того, что число мальчиков, которые родятся в Москве в текущем году, превысит число родившихся девочек не менее, чем на $1500$.

\paragraph{Решение.}\hfill\\
\[
    \Expect X_{childrens} = 122500, \quad \Prob_{man} = 0.51\,.
\]

\noindent
Вычислислим количество мальчиков рождаемых за год

\[
    \Expect X_{man} = \Prob_{man} \cdot\, \Expect X_{childrens} = 0.51 \cdot 122500 = 62475\,.
\]

\noindent
Воспользуемся \emph{первым неравенством Чебышева}, чтобы вычислить вероятность того, что \emph{число мальчиков превысит} число девочек \emph{не менее}, чем на $1500$.


\begin{align*}
    \Prob \{ X \geq \Expect X_{man} + 1500 \} &\leq \frac{\Expect X_{man}}{\Expect X_{man} + 1500};
    \\\\
    \Prob \{ X \geq 62475 + 1500 \} &\leq \frac{62475}{62475 + 1500};
    \\\\
    \Prob \{ X \geq 63975 \} &\leq 0.9765\,.
\end{align*}

\paragraph{Ответ.} $\Prob \{ X \geq \Expect X_{man} + 1500 \} \leq 0.9765$\,.