% !TEX root = main.tex

\section{Задача №4. Доверительные интервалы}

\paragraph{Условие.} По результатам $n = 10$ измерений прибором, не имеющим систематической ошибки, получены следующие отклонения ёмкости конденсатора от номинального значения (пФ)
\[
    5.4,\, -13.9,\, -11.0,\, 7.2,\, -15.6,\, 29.2,\, 1.4,\, -0.3,\, 6.6,\, -9.9\,.
\]
Найти $90\%$-ые доверительные интервалы для среднего значения отклонения ёмкости от номинального значения и её среднего квадратного отклонения.

\paragraph{Решение.}

\[
    \gamma = 0.9, \quad \vec{x}_{10}, \quad n = 10\,.
\]

\noindent
Вычислим среднее значение отклонения ёмкости от номинального
\[
    \overline{x}_{10} = \frac{1}{10} \sum_{i = 1}^{10} x_i = -0.09 
\]

\noindent
Вычислим несмещённое отклонение
\begin{gather*}
    S^2 = \frac{1}{9} \sum_{i = 1}^{10} (x_i - \overline{x}_{10})^2 = 181.639;
    \\
    S = \sqrt{S^2} = \sqrt{181.639} \approx 13.48\,.
\end{gather*}

\noindent
Вычислим одностороннюю область $\alpha$
\[
    \gamma = 1 - 2\alpha \;\Rightarrow\; \alpha = \frac{1 - \gamma}{2} = \frac{1 - 0.9}{2} = 0.05\,.
\]

\noindent
Из таблицы квантилей распределения Стьюдента получим квантиль
\[
    t_{1 - \alpha}(n - 1) = t_{1 - 0.05}(10 - 1) = t_{0.95}(9) = 1.833\,.
\]

\noindent
Вычислим точность интервальной оценки для среднего значения отклонения ёмкости от номинального
\[
    E_1 = t_{1 - \alpha}(n - 1) \frac{S}{\sqrt{n}} = t_{0.95}(9)\frac{S}{\sqrt{n}} = 1.833 \cdot \frac{13.48}{\sqrt{10}} \approx 7.81\,.
\]

\noindent
Построим $90\%$-ый доверительный интервал для среднего значения отклонения ёмкости от номинального
\begin{align*}
    \overline{x}_{10} - E_1 &< \mu < \overline{x}_{10} + E_1;
    \\
    -0.09 - 7.81 &< \mu <  -0.09 + 7.81;
    \\
    -7.9 &< \mu < 7.72\,.
\end{align*}

\noindent
Из таблицы квантилей $\chi^2$-распределения получим квантили
\begin{gather*}
    \chi^2_{\alpha}(n - 1) = \chi^2_{0.05}(10 - 1) = \chi^2_{0.05}(9) = 3.3251;
    \\
    \chi^2_{1 - \alpha}(n - 1) = \chi^2_{1 - 0.05}(10 - 1) = \chi^2_{0.95}(9) = 16.919\,.
\end{gather*}

\noindent
Построим $90\%$-ый доверительный интервал для среднего квадратичного отклонения
\begin{align*}
    \frac{(n-1)S^2}{\chi^2_{1 - \alpha}(n - 1)} &< \sigma^2 < \frac{(n-1)S^2}{\chi^2_{\alpha}(n - 1)};
    \\
    \frac{9 \cdot 181.639}{16.919} &< \sigma^2 < \frac{9 \cdot 181.639}{3.3251};
    \\
    96.62 &< \sigma^2 < 491.64.
\end{align*}

\paragraph{Ответ.} а) $-7.9 < \mu < 7.72$; б) $96.62 < \sigma^2 < 491.64$.