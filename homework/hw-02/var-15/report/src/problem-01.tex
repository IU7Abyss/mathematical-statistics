% !TEX root = main.tex

\section{Задача №1. Проверка параметрических гипотез}

\paragraph{Условие.} До наладки станка была проверена точность изготовления $n_1 = 10$ втулок, в результате чего получено значение $S^2(\overrightarrow{X}_{n_1}) = 9.6\; \text{мкм}^2$. После наладки была проверена партия из $n_2 = 15$ втулок и получено значение $S^2(\overrightarrow{Y}_{n_2}) = 5.7\; \text{мкм}^2$. Считая распределение контролируемого признака нормальным, при уровне значимости $\alpha = 0.05$ проверить гипотезу о том, что после наладки станка точность изготовления втулок увеличилась.

\paragraph{Решение.}\hfill\\
Определим следующие случайные величины учитывая, что по условию распределение \textit{нормальное}
\[
    X \text{ --- точность изготовления некоторой втулки \emph{до наладки} станка}, \quad X \sim N(\mu_1, \sigma_1^2);
\]
\[
    X \text{ --- точность изготовления некоторой втулки \emph{после наладки} станка}, \quad Y \sim N(\mu_2, \sigma_2^2).
\]
Введём следующие гипотезы
\[
    H_0 = \{ \text{точность изготовления втулок после наладки станка \emph{не изменилась}} \} = \{ \sigma_1^2 = \sigma_2^2 \}.
\]
\[
    H_1 = \{ \text{точность изготовления втулок после наладки станка \emph{увеличилась}} \} = \{ \sigma_1^2 > \sigma_2^2 \}.
\]
Построим критическое множество $W$
\begin{equation}
    W = \left\{ (\overrightarrow{X}_{n_1}, \overrightarrow{Y}_{n_2}) \colon T(\overrightarrow{X}_{n_1}, \overrightarrow{Y}_{n_2}) \geq F_{1 - \alpha}(n_1 - 1, n_2 - 1) \right\}.
\end{equation}
Вычислим статистику $T(\overrightarrow{X}_{n_1}, \overrightarrow{Y}_{n_2})$
\begin{equation}
    T(\overrightarrow{X}_{n_1}, \overrightarrow{Y}_{n_2}) = \frac{\max \left\{ S^2(\overrightarrow{X}_{n_1}), S^2(\overrightarrow{Y}_{n_2}) \right\}}{\min \left\{ S^2(\overrightarrow{X}_{n_1}), S^2(\overrightarrow{Y}_{n_2}) \right\}} \sim F_{1 - \alpha}(n_1 - 1, n_2 - 1).
\end{equation}
\begin{equation}
    T(\overrightarrow{X}_{n_1}, \overrightarrow{Y}_{n_2}) = \frac{9.6}{5.7} = 1.684 \sim F_{0.95}(9, 14).
\end{equation}
Из \emph{таблицы квантилей распределения Фишера} $F_{0.95}(9, 14) = 2.65$. Так как $1.684 \ngeq 2.65$, то принимаем гипотезу $H_0$.


\paragraph{Ответ.} $1.684 \ngeq 2.65 \;\Rightarrow\;$ точность изготовления втулок после наладки станка \emph{не изменилась}.