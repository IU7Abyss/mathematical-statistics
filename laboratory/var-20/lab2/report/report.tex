% !TEX root = main.tex

\section{Отчёт}

\subsection{Определения и формулы}

\subsubsection{Определение $\gamma$-доверительного интервала для значения параметра распределения случайной величины}

Пусть $\vec{X}_{n}$ -- случайная выборка объема $n$ из генеральной совокупности $X$ с функцией распределения $F(x;\theta)$, зависящей от параметра 
$\theta$, значение которого неизвестно.

Предположим, что для параметра $\theta$ в построенном интервале $(\underline{\theta}(\vec{X}_{n}),\;\overline{\theta}(\vec{X}_{n}))$, где $\underline{\theta}(\vec{X}_{n})$ и $\overline{\theta}(\vec{X}_{n})$ являются функциями случайной выборки $\vec{X}_{n}$, такими, что выполняется равенство

\[
P\{\underline{\theta}(\vec{X}_{n})<\theta<\overline{\theta}(\vec{X}_{n})\} = \gamma
\]

В этом случае интервал $(\underline{\theta}(\vec{X}_{n}),\;\overline{\theta}(\vec{X}_{n}))$ называют интервальной оценкой для параметра $\theta$ с коэффициентом доверил $\gamma$ (или, сокращенно, $\gamma$ - доверительной интервальной оценкой), а $\underline{\theta}(\vec{X}_{n})$ и $\overline{\theta}(\vec{X}_{n})$ соответственно нижней и верхней границами интервальной оценки. Интервальная оценка $(\underline{\theta}(\vec{X}_{n}),\;\overline{\theta}(\vec{X}_{n}))$ представляет собой интервал со случайными границами, который с заданной вероятностью $\gamma$ накрывает неизвестное истинное значение параметра $\gamma$.

Интервал $(\underline{\theta}(\vec{x}_{n}),\;\overline{\theta}(\vec{x}_{n}))$ называют доверительным интервалом для параметра в с коэффициентом доверия $\gamma$ или $\gamma$ - доверительным интервалом, где $\vec{x}_{n}$ -- любая реализация случайной выборки $\vec{X}_{n}$.

\subsubsection{Формулы для вычисления границ $\gamma$-доверительного интервала}

\paragraph{Нормальное распределение}

Пусть $\vec{X}_{n}$ — случайная выборка объема $n$ из генеральной совокупности $X$, распределенной по нормальному закону с параметрами $\mu$ и $\sigma^{2}$.

\paragraph{Оценка для математического ожидания при известной дисперсии}

\begin{align}
\underline{\mu}(\vec{X}_{n}) &= \overline{X}-\frac{S(\vec{X}_{n})}{\sqrt{n}}t_{1-\alpha}(n-1),\\
\overline{\mu}(\vec{X}_{n}) &= \overline{X}+\frac{S(\vec{X}_{n})}{\sqrt{n}}t_{1-\alpha}(n-1),
\end{align}

\noindent где:
\begin{itemize}
	\item $\overline{X}$ -- оценка мат. ожидания,
	\item $n$ -- число опытов,
	\item $S(\vec{X}_{n})$ точечная оценка дисперсии случайной выборки $\vec{X}_{n}$, $t_{1-\alpha}(n-1)$ квантиль уровня $1-\alpha$ для распределения Стьюдента с $n-1$ степенями свободы, $\alpha$ величина, равная $\frac{(1-\gamma)}{2}$.
\end{itemize}

\paragraph{Оценка для дисперсии}


\begin{align}
\underline{\sigma^{2}}(\vec{X}_{n}) & =\frac{S(\vec{X}_{n})(n-1)}{\chi_{1-\alpha}^{2}(n-1)},\\
\overline{\sigma^{2}}(\vec{X}_{n}) & =\frac{S(\vec{X}_{n})(n-1)}{\chi_{\alpha}^{2}(n-1)},
\end{align}

\noindent где:
\begin{itemize}
	\item $n$ -- объем выборки,
	\item $\chi_{\alpha}^{2}(n-1)$ -- квантиль уровня $\alpha$ для распределения $\chi^{2}$ с $n-1$ степенями свободы,
	\item $\alpha$ -- величина, равная $\frac{(1-\gamma)}{2}$.
\end{itemize}
